\documentclass[11pt,a4paper,oneside]{scrartcl}
\usepackage[utf8]{inputenc}
\usepackage[english,russian]{babel}
\usepackage[top=1cm,bottom=1cm,left=1cm,right=1cm]{geometry}

\begin{document}
\pagestyle{empty}

\begin{center}
{\large\scshape\bfseries Программа курса <<Математическая логика>>}\\
{\large\scshape Вопросы к первому коллоквиуму.}\\
\itshape ИТМО, группы M3232--M3239, осень 2025 г.
\end{center}

%\vspace{0.3cm}

\begin{enumerate}
\item Исчисление высказываний:
\begin{enumerate}
\item Предметный язык и язык исследователя (метаязык). Соглашения об обозначениях. Схемы формул.
\item Язык исчисления высказываний.
\item Оценка высказываний, общезначимость, следование.
\item Доказуемость, гипотезы (контекст), выводимость.
\item Корректность, полнота, противоречивость и непротиворечивость (эквивалентные формулировки).
\item Теорема о дедукции для исчисления высказываний (формулировка). Теорема о полноте исчисления высказываний (формулировка).
\end{enumerate}
\item Топологическое пространство
\begin{enumerate}
\item Определение.
\item Примеры (топология стрелки, Зарисского, топология на деревьях). 
\item Открытые и замкнутые множества. Связность. Компактность. 
\item Непрерывные функции.
\end{enumerate}
\item Гильбертов вывод и натуральный вывод.
\item Интуиционистское исчисление высказываний:
\begin{enumerate}
\item Доказательства чистого существования.
\item BHK-интерпретация. 
\item Закон исключённого третьего, принцип взрыва, связь с КИВ и ИИВ.
\item Решётки. 
\item Дистрибутивная решётка.
\item Псевдодополнение. Булевы и псевдобулевы алгебры.
\item Алгебра Линденбаума. 
\item Полнота интуиционистского исчисления высказываний в псевдобулевых алгебрах (формулировка, идея доказательства).
\item Модели Крипке. Вынужденность.
\item Сведение моделей Крипке к псевдобулевым алгебрам. 
\item Нетабличность ИИВ (формулировка теоремы).
\end{enumerate}
\item Изоморфизм Карри-Ховарда. Интерпретация связок.
\item Дизъюнктивность интуиционистского исчисления высказываний.
\begin{enumerate}
\item Гёделева алгебра. Операция $\Gamma(A)$.
\item Дизъюнктивность ИИВ (формулировка, идея доказательства).
\end{enumerate}
\item Подрешётка. Разрешимость интуиционистского исчисления высказываний (опеределения).

\end{enumerate}
\end{document}
