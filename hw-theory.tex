\documentclass[10pt,a4paper,oneside]{article}
\usepackage[utf8]{inputenc}
\usepackage[english,russian]{babel}
\usepackage{amsmath}
\usepackage{amsthm}
\usepackage{amssymb}
\usepackage{enumerate}
\usepackage{stmaryrd}
\usepackage{cmll}
\usepackage{mathrsfs}
\usepackage[left=2cm,right=2cm,top=2cm,bottom=2cm,bindingoffset=0cm]{geometry}
\usepackage{proof}
\usepackage{tikz}
\usepackage{multicol}
\usepackage{mathabx}
\usepackage{comment}
\usepackage{hyperref}
\usepackage[normalem]{ulem}
\usepackage{cancel}

\makeatletter
\newcommand{\dotminus}{\mathbin{\text{\@dotminus}}}

\newcommand{\@dotminus}{%
  \ooalign{\hidewidth\raise1ex\hbox{.}\hidewidth\cr$\m@th-$\cr}%
}
\makeatother

\usetikzlibrary{arrows,backgrounds,patterns,matrix,shapes,fit,calc,shadows,plotmarks}

\newtheorem{definition}{Определение}
\newtheorem{theorem}{Теорема}
\begin{document}

\begin{center}{\Large\textsc{\textbf{Теоретические домашние задания}}}\\
             \it Математическая логика, ИТМО, М3232-М3239, осень 2025 года\end{center}

\section*{Задание №1. Знакомство с классическим исчислением высказываний.}

\begin{enumerate}
\item Докажите:
\begin{enumerate}
\item $\vdash (A \rightarrow A \rightarrow B) \rightarrow (A \rightarrow B)$
\item $\vdash \neg (A \with \neg A)$
\item $\vdash A \with B \rightarrow B \with A$
\item $\vdash A \vee B \rightarrow B \vee A$
\item $A \with \neg A \vdash B$
\end{enumerate}

\item Докажите:
\begin{enumerate}
\item $\vdash A \rightarrow \neg \neg A$
\item $\neg A, B \vdash \neg(A\& B)$
\item $\neg A,\neg B \vdash \neg( A\vee B)$
\item $ A,\neg B \vdash \neg( A\rightarrow B)$
\item $\neg A, B \vdash  A\rightarrow B$
\end{enumerate}

\item Докажите:
\begin{enumerate}
\item $\vdash (A \rightarrow B) \rightarrow (B \rightarrow C) \rightarrow (A \rightarrow C)$ 
\item $\vdash (A \rightarrow B) \rightarrow (\neg B \rightarrow \neg A)$ \emph{(правило контрапозиции)}
\item $\vdash \neg (\neg A \with \neg B) \rightarrow (A \vee B)$ \emph{(вариант I закона де Моргана)}
\item $\vdash A \vee B \rightarrow \neg(\neg A \with \neg B)$
\item $\vdash (\neg A \vee \neg B) \rightarrow \neg (A \with B)$ \emph{(II закон де Моргана)}
\item $\vdash (A \rightarrow B) \rightarrow (\neg A \vee B)$
\item $\vdash A \with B \rightarrow A \vee B$
\item $\vdash ((A \rightarrow B) \rightarrow A)\rightarrow A$ \emph{(закон Пирса)}
\item $\vdash A \vee \neg A$
\item $\vdash (A \with B \rightarrow C) \rightarrow (A \rightarrow B \rightarrow C)$
\item $\vdash A \with (B \vee C) \rightarrow (A \with B) \vee (A \with C)$ \emph{(дистрибутивность)}
\item $\vdash (A \rightarrow B \rightarrow C) \rightarrow (A \with B \rightarrow C)$
\item $\vdash (A \rightarrow B) \vee (B \rightarrow A)$
\item $\vdash (A \rightarrow B) \vee (B \rightarrow C) \vee (C \rightarrow A)$
\end{enumerate}

%\item Даны высказывания $\alpha$ и $\beta$, причём $\vdash \alpha\rightarrow\beta$ и $\not\vdash\beta\rightarrow\alpha$. 
%Укажите способ построения высказывания $\gamma$, такого, что
%$\vdash\alpha\rightarrow\gamma$ и $\vdash\gamma\rightarrow\beta$, причём $\not\vdash\gamma\rightarrow\alpha$ и
%$\not\vdash\beta\rightarrow\gamma$.

\item Покажите, что если $\alpha \vdash \beta$ и $\neg\alpha\vdash\beta$, то $\vdash\beta$.

\item Давайте вспомним, что импликация правоассоциативна: $\alpha\rightarrow\beta\rightarrow\gamma \equiv \alpha\rightarrow(\beta\rightarrow\gamma)$.
Но рассмотрим иную расстановку скобок: $(\alpha\rightarrow\beta)\rightarrow\gamma$. Возможно ли доказать логическое следствие
между этими вариантами расстановки скобок --- и каково его направление?

\item Возможно ли, что какая-то из аксиом задаётся двумя разными схемами аксиом? Опишите все возможные коллизии для какой-то одной такой пары схем аксиом.
Ответ обоснуйте (да, тут потребуется доказательство по индукции).

\item Заметим, что можно вместо отрицания ввести в исчисление ложь. Рассмотрим \emph{исчисление высказываний с ложью}.
В этом языке будет отсутствовать одноместная связка $(\neg)$, вместо неё будет присутствовать нульместная
связка <<ложь>> $(\bot)$, а 9 и 10 схемы аксиом будут заменены на одну схему:

\begin{tabular}{ll}
$(9_\bot)$ & $((\alpha\rightarrow\bot)\rightarrow\bot)\rightarrow\alpha$
\end{tabular}

Будем записывать доказуемость в новом исчислении как $\vdash_\bot \alpha$, а доказуемость в исчислении высказываний
с отрицанием как $\vdash_\neg \beta$. Также определим операцию трансляции между языками обычного исчисления высказываний и исчисления с ложью
как операции рекурсивной замены $\bot := A \with \neg A$ и $\neg \alpha := \alpha \rightarrow \bot$ (и обозначим их
как $|\varphi|_\neg$ и $|\psi|_\bot$ соответственно).

Докажите:
\begin{enumerate}
\item $\vdash_\bot \alpha$ влечёт $\vdash_\neg |\alpha|_\neg$
\item $\vdash_\neg \alpha$ влечёт $\vdash_\bot |\alpha|_\bot$
\end{enumerate}
\end{enumerate}


\end{document}
